\documentclass[a4paper]{article}

\usepackage{C:/Users/79260/Documents/algorithmic_foundation/settings}


\begin{document}

\textbf{Task H. $2^i + 5^j$ (1 point)}

\SPACE

Print in ascending order all the numbers that can be represented as $2^i + 5^j$ ($i$ and $j$ are non-negative integers) and do not exceed $N$.

\SPACE

\textbf{Input format}

The first line contains the integer $N$ ($0 \le N \le 10^9$).

\SPACE

\textbf{Output format}

Print in ascending order all integers $a$ ($0 \le a \le N$) satisfying the conditions. If a number can be obtained in several ways, then it needs to be displayed as many times.

\SPACE

\noindent \textbf{Sample input:}\\
10\\


\noindent \textbf{Sample output:}\\
2 3 5 6 7 9 9\\


\end{document}
