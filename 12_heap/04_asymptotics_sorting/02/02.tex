\documentclass[a4paper]{article}

\usepackage{../../../settings}


\begin{document}

\textbf{Task F. King of Misopotamia (1 point)}

\SPACE

The king of Misopotamia is a very generous man: when they come to him and ask for help, he gives exactly one of the most expensive gift he has in the treasury. He also has another interesting feature -- he accepts any gift presented to him. But he cannot receive visitors all day, so only $N$ people can get to him per day. For each visitor, the vizier keeps a record. In order not to write a lot, the vizier came up with a system: if a visitor brought a gift in the amount of $S$, then in the protocol he writes $0 \ S$, and if he asks for help, then $1$.

Retrieve the list of King-given gifts according to the protocol.

\SPACE

\textbf{Input format}

The first line contains the number of visitors $N$ ($1 \le N \le 100000$). This is followed by $N$ lines, in the $i$-th of which there is an entry about the $i$-th visitor.

It is guaranteed that at the time of issuing a gift in the treasury there is at least one gift, and the cost of prizes is integers in the range from $0$ to $10^9$.

\SPACE

\textbf{Output format}

For each line of issue it is necessary to print a number in a separate line -- the value of the issued gift.

\SPACE

\noindent \textbf{Sample input:}\\
10\\
0 941\\
0 6077\\
1\\
0 9560\\
1\\
0 12770\\
1\\
0 2117\\
0 1791\\
1\\


\noindent \textbf{Sample output:}\\
6077\\
9560\\
12770\\
2117\\


\end{document}
