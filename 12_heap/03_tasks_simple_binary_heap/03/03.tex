\documentclass[a4paper]{article}

\usepackage{../../../settings}


\begin{document}

\textbf{Task A. Increasing Priority (1 point)}

\SPACE

The maximum heap is given and requests are executed on it.

The query is given by two integers $i$ and $x$. It is required to increase the value of the $i$-th element of the heap by $x$ and perform \textit{SiftUp} to restore the heap.

\SPACE

\textbf{Input format}

The first line contains the heap size $N$ ($1 \le N \le 10^5$).

The second line introduces the heap itself -- $N$ different integers, each of which modulo does not exceed $10^9$. It is guaranteed that these numbers make up the correct maximum heap.

The third line introduces the number $M$ ($0 \le M \le 10^5)$ -- the number of requests.

The following $M$ lines introduce the queries themselves, one per line.

It is guaranteed that $1 \le i \le N$, $x \ge 0$, the new value of the heap element does not exceed $10^9$ and differs from the current values of all other elements of the heap.

\SPACE

\textbf{Output format}

As a response to the query, it is required to display one number in a separate line -- the number of the heap element in which the changed element turned out after \textit{SiftUp}.

In addition, after all the requests have been completed, it is necessary to display the heap in its final state.

\SPACE

\noindent \textbf{Sample input:}\\
6\\
12 6 8 3 4 7\\
2\\
5 11\\
3 6\\


\noindent \textbf{Sample output:}\\
1\\
3\\
15 12 14 3 6 7\\


\end{document}
