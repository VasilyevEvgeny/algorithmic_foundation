\documentclass[a4paper]{article}

\usepackage{../../../settings}


\begin{document}

\textbf{Task <<Naive insertion>>}

You are given a sequence of $n$ distinct non-negative integers. Create an empty binary search tree and insert the given numbers into it one by one.

Insertion should be performed <<naively>>, that is, when inserting, the structure of the tree does not change, and the new element is simply suspended from one of the leaves of the tree, becoming a new leaf.

For the resulting tree, output preorder, inorder, and postorder traversals.

\SPACE

\textbf{Input format}

The first line of the input contains an integer $n$ -- the number of numbers in the sequence ($1 \le n \le 1000$).

The second line contains $n$ different non-negative integers -- the elements of the sequence. Numbers do not exceed $10^9$.

\SPACE

\textbf{Output format}

Output three lines. On the first line print the preorder traversal, on the second line the inorder traversal, and on the third line the postorder traversal of the resulting tree.

\LINE

\noindent \textbf{Sample input:}\\
5\\
6 9 0 1 7\\

\noindent \textbf{Sample output:}\\
6 0 1 9 7\\
0 1 6 7 9\\
1 0 7 9 6\\

\end{document}