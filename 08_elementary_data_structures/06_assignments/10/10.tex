\documentclass[a4paper]{article}

\usepackage{/home/evasilyev/CLionProjects/algorithmic_foundation/settings}


\begin{document}

Implement a <<minimum queue>> data structure. The structure should support the following types of operations: add a number to the end of the queue, remove a number from the beginning of the queue, and output the minimum number in the queue.

The queue is initially empty.

The first line of the input contains an integer $q$ -- the number of operations with the queue $(1 \le q \le 10^6$).

Each of the following $q$ lines contains a description of the operation with the queue:
\begin{itemize}
\item \textbf{push} $\mathbf{x}$ -- add an integer $x$ to the end of the queue ($0 \le x \le 10^9$);
\item \textbf{pop} -- delete the number at the beginning of the queue. It is guaranteed that the queue is not empty;
\item \textbf{min} -- print the minimum number in the queue. It is guaranteed that the queue is not empty.
\end{itemize}


For each min operation print the minimum number in the queue.

\LINE

\noindent \textbf{Sample input:}\\
7\\
push 3\\
push 2\\
min\\
pop\\
min\\
push 1\\
min\\

\noindent \textbf{Sample output:}\\
2\\
2\\
1\\


\end{document}
