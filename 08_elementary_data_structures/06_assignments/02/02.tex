\documentclass[a4paper]{article}

\usepackage{../../../settings}


\begin{document}

Implement a minimum stack data structure. The structure should support the following types of operations: add a number to the top of the stack, remove the number from the top of the stack and print the minimum number on the stack. The stack is initially empty.

The first line of the input contains an integer $q$ - the number of operations with the stack ($1 \le q \le 10^6$).

Each of the following qq lines contains a description of the stack operation:
\begin{itemize}
\item \textbf{push} $x$ -- add an integer $x$ to the top of the stack ($0 \le x \le 10^9$)
\item \textbf{pop} -- delete the number lying on the top of the stack. It is guaranteed that the stack is not empty
\item \textbf{min} -- print the minimum number on the stack. It is guaranteed that the stack is not empty
\end{itemize}
For each min operation, output the minimum number on the stack.

\LINE

\noindent \textbf{Sample input:}\\
7\\
push 3\\
push 2\\
min\\
pop\\
min\\
push 1\\
min\\


\noindent \textbf{Sample output:}\\
2\\
3\\
1\\



\end{document}
