\documentclass[a4paper]{article}

\usepackage{C:/Users/79260/Documents/algorithmic_foundation/settings}


\begin{document}

Given a non-empty, singly linked list with head node, return a middle node of linked list. If there are two middle nodes, return the second middle node.

\ \\

\noindent C++:

\begin{lstlisting}[style=C++]
/**
 * Definition for singly-linked list.
 * struct ListNode {
 *     int val;
 *     ListNode *next;
 *     ListNode(int x) : val(x), next(NULL) {}
 * };
 */
class Solution {
public:
    ListNode* middleNode(ListNode* head) {
        
    }
};
\end{lstlisting}

\LINE

\noindent \textbf{Sample input 1:}\\
$[1, 2, 3, 4, 5]$\\

\noindent \textbf{Sample output 2:}\\
Node 3 from this list (Serialization: [3,4,5])
The returned node has value 3.  (The judge's serialization of this node is [3,4,5]).
Note that we returned a ListNode object ans, such that:
ans.val = 3, ans.next.val = 4, ans.next.next.val = 5, and ans.next.next.next = NULL.

\SPACE

\noindent \textbf{Sample input 1:}\\
$[1, 2, 3, 4, 5, 6]$\\

\noindent \textbf{Sample output 2:}\\
Node 4 from this list (Serialization: [4,5,6])
Since the list has two middle nodes with values 3 and 4, we return the second one.


\end{document}
