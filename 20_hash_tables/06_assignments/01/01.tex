\documentclass[a4paper]{article}

\usepackage{../../../settings}


\begin{document}

\textbf{Task <<Set>>}

Petya launches another startup. The business idea, as usual with Petya, is simple, brilliant, and promises a lot of profit. You work for Petya as a programmer, which means that you have to implement all his crazy ideas.

Today's idea is an assistant app for mathematicians. It allows you to store a set of integers and modify it. The app prototype will have only three functions:
\begin{itemize}
\item Add an integer $x$ to the set. If this number is already in the set, the set does not change;
\item Remove an integer $x$ from the set. If this number is not in the set, the set does not change;
\end{itemize}

Calculate the sum of all numbers in the set. Initially, the set is empty. Help Petya get rich by implementing an application prototype that supports these operations.

\SPACE

\textbf{Input format}

The first line of the input contains a single integer $n$ -- the number of requests ($1 \le n \le 5 \cdot 10^5$).

Each of the next $n$ lines contains a description of the next request. First, in the description there is an integer $t$ -- the request type ($1 \le t \le 3$):
\begin{itemize}
\item If $t = 1$, then an integer $x$ follows, and this is a request to add $x$ to the set;
\item If $t = 2$, then an integer $x$ follows, and this is a request to remove the number $x$ from the set;
\item If $t = 3$, then this is a request to calculate the sum of numbers in the set.
\end{itemize}

All added and removed numbers do not exceed $10^9$ in absolute value.

\SPACE

\textbf{Output format}

For each query of the third type, print the sum of the numbers in the set on a separate line.

\LINE

\noindent \textbf{Sample input:}\\
7\\
1 3\\
1 4\\
3\\
1 3\\
3\\
2 4\\
3\\

\noindent \textbf{Sample output:}\\
7\\
7\\
3\\

\end{document}