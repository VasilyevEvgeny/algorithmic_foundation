\documentclass[a4paper]{article}

\usepackage{../../../settings}


\begin{document}

In the center of city Che there is a pedestrian street -- one of the most popular places for residents walking in the city. It is very nice to walk along this street, because $n$ funny monuments are located along the street.

The girl Masha from the city of Che likes two boys from her school, and she can't make a choice between them. In order to make a final decision, she decided to date both boys at the same time. Masha wants to choose two monuments on a pedestrian street, near which the boys will wait for her. At the same time, she wants to choose such monuments so that the boys do not see each other. Masha knows that because of the fog, the boys will see each other only if they are at a distance of no more than $r$ meters.

Masha became interested, how many ways there are to choose two different monuments for organizing dates.

The first line of the input contains two integers $n$ and $r$ ($2 \le n \le 300$, $1 \le r \le 10^9$) -- the number of monuments and the maximum distance at which the boys can see each other.

The second line contains $n$ positive numbers $d_1, \ldots, d_n$, where $d_i$ is the distance from the $i$-th monument to the beginning of the street. All monuments are at different distances from the beginning of the street. Monuments are given in increasing order of distance from the beginning of the street ($1 \le d_1 < d_2 < \ldots < d_n \le 10^9$).

Print one number -- the number of ways to choose two monuments for organizing dates.

\LINE

\noindent \textbf{Sample input:}\\
4 4\\
1 3 5 8\\


\noindent \textbf{Sample output 1:}\\
2


\end{document}
