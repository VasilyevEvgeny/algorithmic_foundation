\documentclass[a4paper]{article}

\usepackage{C:/Users/79260/Documents/algorithmic_foundation/settings}


\begin{document}

The first line contains two numbers: $N$ and $X$. The next line contains $N$ natural numbers not exceeding 10000.

In the output stream, print the boundaries of the segment of the input array whose sum of elements is $X$ (the indices in the array are numbered from 1). If there are several such segments, print any of them. If there is no such segment, print <<1 0>>.

It is known that $1 \le N \le 10^5$.

\LINE

\noindent \textbf{Sample input 1:}\\
7 13\\
3 1 5 6 2 8 7\\


\noindent \textbf{Sample output 1:}\\
3 5


\SPACE

\noindent \textbf{Sample input 2:}\\
7 18\\
3 1 5 6 2 8 7\\


\noindent \textbf{Sample output 2:}\\
1 0


\SPACE

\noindent \textbf{Sample input 3:}\\
10 1\\
1 2 3 4 5 6 7 8 9 10\\


\noindent \textbf{Sample output 3:}\\
1 1

\SPACE


\noindent \textbf{Sample input 4:}\\
10 3\\
1 2 3 4 5 6 7 8 9 10\\


\noindent \textbf{Sample output 4:}\\
1 2


\end{document}
