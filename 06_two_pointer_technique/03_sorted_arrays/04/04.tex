\documentclass[a4paper]{article}

\usepackage{../../../settings}


\begin{document}

Given two sorted arrays of integers and an integer $X$. Find if there are a couple of elements in these two arrays whose sum is $X$.

The first line of input contains three integers $N$, $M$, $X$: the sizes of the first and second array and the required sum, respectively. It is known that $1 \le N, M \le 10^5$, $|X| < 20000$.

The second line of input contains $N$ integers modulo not exceeding 10000, -- the first array.

The third line of input contains $M$ integers, modulo not exceeding 10000, -- the second array.

In the output stream print a pair of numbers that specify the indices of the first and second arrays. The sum of the elements at these indices must be equal to $X$. Indexes are numbered from one. If there are several such pairs, print any. If there is no such pair, print <<0 0>>.

\LINE

\noindent \textbf{Sample input 1:}\\
3 3 4\\
0 2 5\\
-3 -1 2\\


\noindent \textbf{Sample output 1:}\\
2 3


\SPACE

\noindent \textbf{Sample input 2:}\\
3 3 24\\
0 2 5\\
-3 -1 2\\


\noindent \textbf{Sample output 2:}\\
0 0



\end{document}
