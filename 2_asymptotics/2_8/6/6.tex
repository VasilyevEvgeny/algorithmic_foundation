\documentclass[a4paper]{article}

\usepackage{/home/evasilyev/CLionProjects/algorithmic_foundation/settings}


\begin{document}

Рассмотрим следующую функцию, которая раскладывает целое число $x$ на простые множители. Пусть число $x$ не превышает $C$. Оцените асимптотику времени работы этой функции.

\SPACE

\noindent C++:

\begin{lstlisting}[style=C++]
#include <vector>

std::vector<int> primes(int x) {
    std::vector<int> result;
    int i = 2;
    while (i * i <= x) {
        while (x % i == 0) {
            result.push_back(i);
            x /= i;
        }
        ++i;
    }
    if (x != 1) {
        result.push_back(x);
    }
    return result;
}
\end{lstlisting}

\SPACE

\noindent Python:

\begin{lstlisting}[style=Python]
def primes(x):
    result = []
    i = 2
    while i * i <= x:
        while x % i == 0:
            result.append(i)
            x //= i
        i += 1
    if x != 1:
        result.append(x)
    return result
\end{lstlisting}

\SPACE

\LINE

\SPACE

\begin{itemize}
\item $O(\sqrt{C} + \log C)$
\item $O(\sqrt{C} \log C)$
\item $O(\log C)$
\item {\color{red}{$\mathbf{O(\sqrt{C})}$}}
\item $O(C)$
\end{itemize}

\SPACE

\LINE

\SPACE

В худшем случае алгоритм не заходит во внутренний цикл и итерируется во внешнем $\sqrt{n}$ раз. В общем случае $x$ уменьшается логарифмически, сокращая сложность.

\end{document}