\documentclass[a4paper]{article}

\usepackage{../../../settings}


\begin{document}

\textbf{Task <<$K$-th zero>>}

Implement an efficient data structure that allows you to change the elements of an array and calculate the index of the $k$-th zero from the left in a given segment in the array.

\SPACE

\textbf{Input format}

The first line of the input contains an integer $N$ ($1 \le N \le 200'000$) -- the number of numbers in the array.

The second line contains $N$ numbers from $0$ to $100'000$ -- array elements.

The third line contains an integer $M$ ($1 \le M \le 200'000$) -- the number of requests.

Each of the following $M$ lines is a description of the request. First comes a single letter that encodes the type of request:
\begin{itemize}
\item <<s>> -- calculate the index of the $k$-th zero;
\item <<u>> -- update the value of the element.
\end{itemize}

Three numbers follow the <<s>> -- the left and right ends of the segment and the number $k$ ($1 \le k \le N$).

The <<u>> is followed by two numbers -- the element number and its new value.

\SPACE

\textbf{Output format}

For each <<s>> query, print the answer to that query. Print all numbers on one line separated by spaces. If there is no required number of zeros in the requested segment, print $-1$ for this query.

\LINE

\noindent \textbf{Sample input:}\\
5\\
0 0 3 0 2\\
3\\
u 1 5\\
u 1 0\\
s 1 5 3\\

\noindent \textbf{Sample output:}\\
4\\

\end{document}