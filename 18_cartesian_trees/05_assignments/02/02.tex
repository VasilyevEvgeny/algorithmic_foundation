\documentclass[a4paper]{article}

\usepackage{../../../settings}


\begin{document}

\textbf{Task <<Binary search tree>>}

Implement a balanced binary search tree.

You can use a built-in data structure such as std::set for this problem, but we recommend that you write your own implementation of a balanced binary search tree for practice purposes. The code you wrote will be useful in subsequent tasks.

\SPACE

\textbf{Input format}

The input to your program is a description of tree operations, their number does not exceed $100'000$. Each line contains one of the following operations:
\begin{itemize}
\item <<insert $x$>> -- add key $x$ to the tree. If the key $x$ is already in the tree, then nothing needs to be done;
\item <<delete $x$>> -- delete key $x$ from the tree. If the $x$ key is not in the tree, then nothing needs to be done;
\item <<exists $x$>> -- if the key $x$ exists in the tree, print <<true>>, otherwise <<false>>;
\item <<next $x$>> -- print the minimum element in the tree that is strictly greater than $x$, or <<none>> if there is none;
\item <<prev $x$>> -- print the maximum element in the tree strictly less than $x$, or <<none>> if there is none.
\end{itemize}

All numbers in the input file are integers and modulo does not exceed $10^9$.

\SPACE

\textbf{Output format}

Output sequentially the result of all operations exists, next, prev. Print the result of each operation on a separate line.

\LINE

\noindent \textbf{Sample input:}\\
insert 2\\
insert 5\\
insert 3\\
exists 2\\
exists 4\\
next 4\\
prev 4\\
delete 5\\
next 4\\
prev 4\\

\noindent \textbf{Sample output:}\\
true\\
false\\
5\\
3\\
none\\
3\\

\end{document}