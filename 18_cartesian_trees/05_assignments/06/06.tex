\documentclass[a4paper]{article}

\usepackage{../../../settings}


\begin{document}

\textbf{Task <<Operating systems>>}

Vasin's hard drive consists of $M$ sectors.

Vasya sequentially installed various operating systems on it using the following method: he created a new disk partition from successive sectors, starting from the sector with the number $a_i$ and up to sector number $b_i$ inclusive, and installed another system on it.

Moreover, if the next section at least in one sector intersects with some previously created section,
then the previously created partition is <<overwritten>>, and the operating system that was installed on it,
can no longer be loaded.

Write a program that, based on information about which partitions Vasya created on the disk, will determine how many operating systems were eventually installed and are currently running on Vasya's computer.

\SPACE

\textbf{Input format}

The first line of the input contains two integers $M$ and $N$ -- the number of sectors on the hard disk and the number of partitions that Vasya created sequentially ($1 \le M \le 10^9$; $0 \le N \le 100'000$). Next come $N$ pairs of integers $a_i$ and $b_i$, specifying the numbers of the beginning and end sectors of the partition ($1 \le a_i \le b_i \le M$).

\SPACE

\textbf{Output format}

Print a single number -- the number of running operating systems on Vasya's computer.

\LINE

\noindent \textbf{Sample input:}\\
10\\
3\\
1 3\\
4 7\\
3 4\\

\noindent \textbf{Sample output:}\\
1\\

\end{document}