\documentclass[a4paper]{article}

\usepackage{../../../settings}


\begin{document}

\textbf{Task <<Geometry>>}

Implement a data structure that stores a set of points on a plane with rectangular Cartesian coordinates and handles two types of queries:
\begin{itemize}
\item Add point with coordinates $(x, y)$. If a point with such coordinates has already been added, the set does not change;
\item Check whether inside or the border of a rectangle with sides parallel to the coordinate axes and opposite angles at points $(0, 0)$ and $(x, y)$ contains at least one of the added points.
\end{itemize}

Initially, the set of points is empty.

\SPACE

\textbf{Input format}

The first line of the input contains an integer $N$ ($1 \le N \le 100'000$) -- the number of requests.

Each of the following $N$ lines contains a description of the next request: three integers $t$, $x$ and $y$ -- the request type and point coordinates ($1 \le t \le 2$; $1 \le x, y \le 10^9$).

If $t = 1$, then this is a request to add a point $(x, y)$.

If $t = 2$, then this is a request to check if one of the added points exists in a rectangle with corners at points $(0, 0)$ and $(x, y)$.

\SPACE

\textbf{Output format}

For each query of the second type, print <<YES>> if at least one of the added points lies inside or on the border of the given rectangle, and <<NO>> otherwise.

\LINE

\noindent \textbf{Sample input:}\\
5\\
1 2 2\\
2 3 3\\
1 4 4\\
2 5 1\\
2 5 5\\

\noindent \textbf{Sample output:}\\
YES\\
NO\\
YES\\

\end{document}