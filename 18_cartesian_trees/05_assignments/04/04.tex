\documentclass[a4paper]{article}

\usepackage{../../../settings}


\begin{document}

\textbf{Task <<Sum>>}

Implement a data structure that stores a set $S$ of integers that is allowed to perform the following operations:
\begin{itemize}
\item add(i) -- add the number ii to the set $S$ (if it is already there, then the set does not change);
\item sum(l, r) -- print the sum of all elements $x$ from $S$ that satisfy the inequality $l \le x \le r$.
\end{itemize}

The set $S$ is initially empty.

\SPACE

\textbf{Input format}

The first line of the input contains an integer $n$ -- the number of operations ($1 \le n \le 300'000$).

The next $n$ lines contain descriptions of operations. Each operation is either <<+ $i$>> or <<? $l$ $r$>>.

Operation <<? $l$ $r$>> specifies the query $sum(l, r)$.

If the <<+ $i$>> operation is the first of all operations, or comes immediately after another <<+>> operation, then it specifies an $add(i)$ operation.

If it comes immediately after the query <<?>>, and the result of this query was $y$, then the operation is performed $add ((i + y) \bmod 10^9)$).

In all queries and adding operations, the parameters are in the range from $0$ to $10^9$.

\SPACE

\textbf{Output format}

For each query print one number -- the answer to the query.

\LINE

\noindent \textbf{Sample input:}\\
6\\
+ 1\\
+ 3\\
+ 3\\
? 2 4\\
+ 1\\
? 2 4\\

\noindent \textbf{Sample output:}\\
3\\
7\\

\end{document}