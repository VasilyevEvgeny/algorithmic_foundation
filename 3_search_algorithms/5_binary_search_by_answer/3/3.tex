\documentclass[a4paper]{article}

\usepackage{/home/evasilyev/CLionProjects/algorithmic_foundation/settings}


\begin{document}

Along the route $N$ km long, there are waypoints every kilometer. Near the zero pillar, as well as near the pillar with the number $N$, there are roadside cafes. In addition, roadside cafes are also located near some other waybills.

We want to place new roadside cafes on the $K$ route so that the distance between any two neighboring cafes does not exceed $M$ km.

The first line of the standard input contains integers $N$, $K$, $M$ -- the length of the route in kilometers, the number of new roadside cafes and the maximum permissible distance between two neighboring cafes after the appearance of new ones ($10 \le N \le 1000$, $1 \le K \le 100$, $1 \le M \le N$). 

The following is given an integer $L$ followed by $L$ of natural numbers -- the numbers of road poles that already have roadside cafes (in addition to the two extreme ones). One pillar has no more than one cafe. It is guaranteed that $L + K < N$.

In the output stream, output <<YES>>, if where there is no cafe now, you can build $K$ new roadside cafes so that the distance between any two neighboring cafes does not exceed $M$ km. Otherwise, print <<NO>>.

\SPACE

\textbf{Hint}

How can you apply the greedy algorithm in this task?

\SPACE

\textbf{Comment}

When you solve this problem, check out the author's solution on the solution forum -- it is used in the next video.


\LINE

\noindent \textbf{Sample input 1:}\\
15 2 3\\
3\\
6 3 12\\

\noindent \textbf{Sample output 1:}\\
YES

\SPACE

\noindent \textbf{Sample input 2:}\\
15 2 1\\
3\\
6 3 12\\

\noindent \textbf{Sample output 2:}\\
NO


\end{document}
