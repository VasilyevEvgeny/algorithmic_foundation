\documentclass[a4paper]{article}

\usepackage{../../../settings}


\begin{document}

Along the route $N$ km long, there are waypoints every kilometer. Near the zero pillar, as well as near the pillar with the number $N$, there are roadside cafes. In addition, roadside cafes are also located near some other waybills. We want to place new roadside cafes on the $K$ route so that the maximum distance between any two neighboring cafes is minimal.

The first line of the standard input contains integers $N$ and $K$ - the length of the route in kilometers and the number of new roadside cafes ($10 \le N \le 1000$, $1 \le K  \le 1000$).

The following is given an integer $L$ followed by $L$ of natural numbers - the numbers of road poles that already have roadside cafes (in addition to the two extreme ones). One pillar has no more than one cafe. It is guaranteed that $L + K < N$.

In the output stream print an integer - the maximum distance between two neighboring cafes after we build $K$ new ones.

\LINE

\noindent \textbf{Sample input 1:}\\
14 4\\
2\\
4 10\\

\noindent \textbf{Sample output 1:}\\
2

\SPACE

\noindent \textbf{Sample input 2:}\\
14 3\\
2\\
4 10\\

\noindent \textbf{Sample output 2:}\\
3

\SPACE

\noindent \textbf{Sample input 3:}\\
1000 1\\
3\\
300 701 800\\

\noindent \textbf{Sample output 3:}\\
300

\end{document}
