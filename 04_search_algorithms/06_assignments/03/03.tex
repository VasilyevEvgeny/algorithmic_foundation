\documentclass[a4paper]{article}

\usepackage{../../../settings}


\begin{document}

$n$ meerkats stand in ascending order of growth. We know the growth of each of them. Some of them burrow into the ground, some climb onto the mounds, but the growth is considered to be the distance from the ground to the back of the head of a meerkat, so some may have negative growth. So they cost $k$ days. Every day one marmot comes to them. The growth of the groundhog is measured and the groundhog is interested in what place he would be placed in a row (at the same time, of course, they will not put him there, he is a groundhog, after all, and he is interested only out of idle curiosity). If its growth coincides with the growth of one or more meerkats, then the groundhog is interested in the smallest possible position. Help meerkats answer stupid groundhog questions. By the way, meerkats number themselves from scratch.

The first line of the input contains two numbers $n$ and $k$ ($1 \le n, k \le 100001$). The second line contains $n$ integers - the growth of each meerkat in order. The third line contains $k$ integers - the growth of each arriving groundhog. All given numbers in absolute value do not exceed $10^9$.

For each of the $k$ marmots, output the corresponding positions in the ranks (each in a separate line).
\LINE

\noindent \textbf{Sample input 1:}\\
5 5\\
1 3 5 7 9\\
2 4 8 1 6

\noindent \textbf{Sample output 1:}\\
1\\
2\\
4\\
0\\
3

\SPACE

\noindent \textbf{Sample input 2:}\\
6\\
1 5 4 3 2 12\\

\noindent \textbf{Sample output 2:}\\
4

\end{document}
