\documentclass[a4paper]{article}

\usepackage{../../../settings}


\begin{document}

In a certain state, a year consists of $N$ days, and the week consists of $\omega$ consecutive working days. Unions require that $h$ days off be added after ww business days, so the week will consist of $\omega + h$ days - first $\omega$ business days, then $h$ weekends, again $\omega$ business days, then $h$ weekends, etc. The last week may be incomplete (if $N$ is not divisible by $\omega + h$), then in the incomplete week, business days first go (no more than $\omega$), then the weekend.

The oligarchs agree to such a reform of the calendar, but insist that the total number of working days in a year be at least $M$. Determine what the maximum number of days off in each week can be added to the calendar so as to fulfill the requirements of the oligarchs.

The program receives three integers $N$, $M$, $\omega$, written in one line, separated by spaces. $N$ is the number of days in a year, $M$ is the minimum total number of working days in a year, $\omega$ is the number of working days in one week. $1 \le N \le 10^{18}$, $1 \le M \le N$, $1 \le \omega < M$.

The program should print a single integer - the maximum number of days off that can be added per week so that the total number of working days in a year will be at least $M$.

\LINE

\noindent \textbf{Sample input:}\\
100 70 8\\

\noindent \textbf{Sample output:}\\
3

\end{document}
