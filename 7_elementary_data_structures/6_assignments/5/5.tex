\documentclass[a4paper]{article}

\usepackage{/home/evasilyev/CLionProjects/algorithmic_foundation/settings}


\begin{document}

There is a very long fence in front of Seryozha’s house. Right now, workers came to the fence and started painting it very quickly.

A fence is a sequence of $10^9$ boards numbered from $1$ to $10^9$. Initially, all boards are color $0$.

Painting takes place in several stages. Each stage is characterized by two numbers $r$ and $c$ . During this step, the first $r$ of the fence boards (numbered $1$ to $r$) are repainted in the color $c$.

Sometimes Seryozha becomes interested in what color one of the planks of the fence is currently painted on. Unfortunately, Seryozha does not always have time to monitor what is happening with the fence, so you will have to answer his questions.

The first line of the input contains one $q$ ($1 \le q \le 100000$) -- the number of events.

The following $q$ lines describe events. Each line begins with a number $t$ -- type of event.
If $t = 1$, then the next stage of painting the fence has passed. After that, two integers $r$ and $c$ follow ($1 \ le r, c \le 10^9$). At this moment, the first $r$ of the fence boards are repainted in $c$ color.
If $t = 2$, then Seryozha wants to know the color of one of the boards. This is followed by a single integer $x$ ($1 \le x \le 10^9$). At this point, Seryozha wants to know the current color of the board with the number $x$.

For each event of the second type print the color of the board that interests Seryozha.

\LINE

\noindent \textbf{Sample input 1:}\\
3\\
1 5 1\\
2 3\\
2 6\\


\noindent \textbf{Sample output 1:}\\
1\\
0\\


\SPACE


\noindent \textbf{Sample input 2:}\\
4\\
1 5 1\\
2 3\\
1 4 2\\
2 3\\


\noindent \textbf{Sample output 2:}\\
1\\
2\\



\end{document}
