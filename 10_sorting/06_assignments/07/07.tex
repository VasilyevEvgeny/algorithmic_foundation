\documentclass[a4paper]{article}

\usepackage{../../../settings}


\begin{document}

Vasya wrote a large number on a long strip of paper and decided to show off this achievement to his older brother Petya. But as soon as he left the room to call his brother, his sister Katya ran into the room and cut a strip of paper into several pieces. As a result, each part contained one or more successive numbers.

Now Vasya cannot remember exactly what number he wrote. Just remember that it was very big. To console his younger brother, Petya decided to find out what the maximum number could be written on a strip of paper before cutting. Help him!

The first line of the input contains an integer $N$ -- the number of pieces of paper strip ($1 \le N \le 100$). Each of the next $N$ lines contains a sequence of digits written on the next part. Each line contains from $1$ to $100$ digits. It is guaranteed that the first digit in at least one line is different from zero.

Print one line -- the maximum number that could be written on the strip before cutting.

\LINE

\noindent \textbf{Sample input:}\\
4\\
2\\
20\\
004\\
66\\


\noindent \textbf{Sample output:}\\
66220004


\end{document}