\documentclass[a4paper]{article}

\usepackage{../../../settings}

\begin{document}

Given an array containing $n$ integers. You need to find the $k$-th order statistics in this array, that is, the element that, after sorting the array in non-descending order, will be at the $k$-th place from the beginning of the array (element indexing starts from zero).

The only line of the input contains three integers $n$, $a_0$ and $k$ are the number of elements in the array, the value of the element with index $0$, and the number of the required order statistics ($1 \le n \le 2 \cdot 10^7$; $0 \le a_0 < 2^{31}$; $0 \le k < n$).

The remaining elements of the array must be generated. Array elements are set using a pseudo-random generator according to the formula: $a_i = (1103515245 \cdot a_{i-1} + 12345) \bmod 2^{31}$. To fill array elements with initial values, you can use the following C++ function:

\lstinputlisting[language=C++, style=CodeStyle]{function.cpp}

\noindent The program should output one integer -- the $k$-th minimum in the given sequence.

\LINE

\noindent \textbf{Sample input:}\\
5 123456789 2\\


\noindent \textbf{Sample output:}\\
850994577



\end{document}