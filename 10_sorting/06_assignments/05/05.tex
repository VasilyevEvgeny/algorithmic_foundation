\documentclass[a4paper]{article}

\usepackage{../../../settings}


\begin{document}

The system administrator remembered that he had not archived user files for a long time. However, the amount of disk space it can store the archive on may be less than the total size of the archived files.

It is known how much space each user's files occupy.

Write a program that, given information about users and free space on the archive disk, will determine the maximum number of users whose data can be archived.

The first line of the input contains an integer $S$ -- the amount of free disk space ($0 \le S \le 10^7$) and the number $N$ is the number of users ($1 \le N \le 10^5$). Then there are $N$ natural numbers not exceeding $300$ -- the amount of data for each user, each written in a separate line.

Print the maximum number of users whose data can be archived.

\LINE

\noindent \textbf{Sample input:}\\
100 2\\
200\\
50\\


\noindent \textbf{Sample output:}\\
1


\end{document}