\documentclass[a4paper]{article}

\usepackage{/home/ev/repos/algorithmic_foundation/settings}


\begin{document}
After a protracted meeting, the director of the firm decided to order a taxi to take the employees home. He ordered $N$ cars -- exactly as many as he has employees. But when they arrived, it turned out that each taxi driver has his own tariff for $1$ kilometer.

Each employee told the director how many kilometers he needed to drive to get home. Different employees must get into different taxis. Now the director wants to determine which of the employees in which taxi should go home so that the total cost of a taxi (and they are borne by the company) is minimal.

The first line of the input contains the number $N$ ($1 \le N \le 1000$) -- the number of company employees (which is the same as the number of taxis called). The second line contains $N$ numbers specifying the distance in kilometers from work to the homes of the company's employees (the first number is for the first employee, the second for the second, and so on). All distances are positive integers not exceeding $1000$. The next line contains $N$ numbers -- fares for one kilometer in a taxi (the first number is in the first taxi car, the second is in the second, and so on). Tariffs are expressed as positive integers not exceeding $10 \ 000$.

Print a single number -- the minimum cost required to drive all the employees home.

\LINE

\noindent \textbf{Sample input:}\\
3\\
10 20 30\\
50 20 30\\


\noindent \textbf{Sample output:}\\
1700

\end{document}