\documentclass[a4paper]{article}

\usepackage{C:/Users/79260/Documents/algorithmic_foundation/settings}


\begin{document}
A group of schoolchildren came to the rink. There are many different sizes of skates for rent. A student may wear skates of any size that is not smaller than the size of his feet. The sizes of all skates and the sizes of schoolchildren's feet are known. Determine the maximum number of students who can go for a ride at the same time.

The first line of the input contains the number $N$ -- the number of skates for rent ($1 \le N \le 10^5$). The second line contains $N$ numbers -- the sizes of the skates. The third line contains the number $M$ -- the number of students in the group ($1 \le M \le 10^5$). The fourth line contains the measurements of their feet. Sizes of skates and legs are natural numbers not exceeding $100$.

\LINE

\noindent \textbf{Sample input:}\\
4\\
41 40 39 42\\
3\\
42 41 42\\


\noindent \textbf{Sample output:}\\
2

\end{document}