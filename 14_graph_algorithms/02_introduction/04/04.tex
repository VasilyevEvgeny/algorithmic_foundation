\documentclass[a4paper]{article}

\usepackage{../../../settings}


\begin{document}

A vertex of a directed graph is called a source if no edges enter into it, and a sink if no edges come out of it.

The directed graph is given by the adjacency matrix. Find all of its source peaks and all sink peaks.

The first line of the input contains the number $N$ -- the number of vertices in the graph ($1 \le N \le 100$), then the adjacency matrix is written -- $N$ rows with $N$ numbers, each of which is $0$ or $1$.

In the first line print $K$ -- the number of sources in the graph, then the numbers of vertices that are sources in ascending order. In the second line, output information about sinks in the same format.

\LINE

\noindent \textbf{Sample input:}\\
5\\
0 0 0 0 0\\
0 0 0 0 1\\
1 1 0 0 0\\
0 0 0 0 0\\
0 0 0 0 0\\

\noindent \textbf{Sample output:}\\
2 3 4\\
3 1 4 5\\



\end{document}
